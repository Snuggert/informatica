\documentclass[pdftex,12pt,a4paper]{article}

\usepackage{wrapfig,amssymb,amsmath,graphicx, subfigure}
\usepackage[dutch]{babel}
\usepackage[top=0.5in, bottom=0.5in, left=1in, right=1in]{geometry}
\pagenumbering{arabic}
\newcommand{\HRule}{\rule{\linewidth}{0.5mm}}

\begin{document}
\begin{titlepage}

% Upper part of the page
\begin{flushleft}
\includegraphics[trim=23mm 0mm 0mm 0mm, width=1\textwidth]{./logo.jpg}\\[1cm] \end{flushleft}
\begin{center}
	\textsc{\Large Reflectie op de Digitale Cultuur}\\[0.5cm]

    % Title
    \HRule \\[0.4cm] { \huge \bfseries Geosociale dating-apps}\\[0.4cm]

    \HRule \\[1.5cm]
    RDC-W3

    % Author and supervisor
\begin{minipage}{0.4\textwidth}
\begin{flushleft} \large \emph{Authors:}\\
Abe \textsc{Wiersma}\\
\end{flushleft}
\end{minipage}
\begin{minipage}{0.4\textwidth} \begin{flushright} \large \end{flushright}\end{minipage}

    \vfill

    % Bottom of the page 
    {\large \today}

\end{center}
\end{titlepage}
\pagebreak
\begin{enumerate}
    \item
        \begin{enumerate}
            \item
            	$$U=\{1,2,3,4,5,6\}$$
            \item
            	gooi 1 heeft:
            		$$U=\{1,2,3,4,5,6\}$$
            	gooi 2 heeft ook:
            		$$U=\{1,2,3,4,5,6\}$$
            	Dit is ook te zien als:
            		$$U=\{2,3,4,5,6,7,8,9,10,11,12\}$$
            	Waarbij de getallen de optelling van het aantal ogen van de twee worpen is.
            \item
            	De kans om tweemaal 6 te gooien is tweemaal $1/6 \rightarrow 1/36$
            \item
            	De kans om met tweemaal gooien 9 ogen te gooien is 4 keer 1/36, namelijk door 4,5 en 6,3 en omgedraaid.
            \item
            	Laat U wederom bestaan uit de getallen 2 t/m 12.
            	Dan is de kans $P(X=i) = \sum_{i=1}^{10}$
            \item
            	Met 2 dobbelstenen kan even ogen gegooid worden door met allebei de dobbelstenen even te gooien of door allebei met de dobbelstenen oneven te gooien, omdat er 1/2 kans is met iedere dobbelsteen even of oneven te gooien heb je 1/2 kans om even te gooien met 2 dobbelstenen.
        \end{enumerate}

    \item
    	$$P(A \wedge B) = 0.4$$
    	$$P(A/B) = 0.1$$
    	$$P(B/A) = 0.3$$
    	$$P((B \vee A)^c) = 0.2$$

	\item
    	\begin{enumerate}
    		\item
    			$$P(A \vee B) = P(A) + P(B) - P(A \wedge B)$$
    			$$P(A \vee B) = P(A) + P(B) - (P(A) + P(B) - P(A \vee B))$$
    			$$P(A \vee B) = P(A \vee B)$$
    		\item
    			$$P(\neg A) = 1 - P(A)$$
    			$$P(\neg A) + P(A) = 1$$
    			$$P(U) = 1$$

    		\item

    	\end{enumerate}
    \item
    	4
    \item 
    	5
\end{enumerate}

\end{document}
