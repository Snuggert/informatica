\documentclass[pdftex,12pt,a4paper]{article}

\usepackage{wrapfig,amssymb,amsmath,graphicx, subfigure}
\usepackage[dutch]{babel}
\usepackage[top=0.5in, bottom=0.5in, left=1in, right=1in]{geometry}
\pagenumbering{arabic}
\newcommand{\HRule}{\rule{\linewidth}{0.5mm}}

\begin{document}
\bibliographystyle{plain}
\noindent Opdracht RDC3\\ 
Abe Wiersma(10433120)\\
\today
\section*{RDC4}
\subsection{De mate van technologische verandering}
In J. David Bolter's essay Turing's man schrijft hij over de opkomst van de computer als entiteit in de samenleving. Een entiteit die evenals de mens een goede en een slechte zijde zal hebben. In de tijd rond het essay was de automatisatie volop aan de gang en werden de simpele repetitieve taken volop overgenomen door de computers. Toch had Bolter's het idee dat in de decades die zouden volgen de techniek van computers zo zou verbeteren dat de kunde van de computers voorbij het repetitieve zou kunnen gaan en dat de computers 'slim' zouden kunnen worden. In het overnemen van de repetitieve taken ziet Bolter ook de afhankelijkheid van de computers ontstaan en ik zou niet anders kunnen concluderen dan dat die afhankelijkheid sinds zijn voorspelling de waarheid is geworden. Ook maakt hij een stuk bredere toekomstvoorspelling waarin hij vrijwel precies het wereldbeeld schetst zoals waarin wij leven: Een wereld waarin alles via computers gaat en waarin elke laag van de samenleving over een computer beschikt.
\subsection{De computer als definiërende technologie}
Bolter spreekt over de computer als de vinding die de $21^{e}$ eeuw definieert en dat er weinig technologieën zijn die een dergelijke status verwerven binnen de geschiedenis van de moderne mens.

\subsection{Turing's man}
De theorie gaat de praktijk vaak voor en bij de computer werd de theorie aangedragen door A.M. Turing. In 1936 publiceerde hij een essay waarin hij de limitaties van computers definieerde. Dit was nog voordat er één enkele programmeerbare computer was gebouwd. In dit essay, dat alleen te lezen was door specialisten in de wiskunde, beschreef Turing ook hoe een 'computer' de menselijke intelligentie perfect zou moeten kunnen imiteren. Hier komt ook het informatica vakgebied Artificial Intelligence vandaan. Volgens Turing zou het zelfs zo zijn dat er in het jaar 2000 computers zouden bestaan die dat al konden doen.
Bolter stelt dat door Turing's beeld van de computer Turing alles wat mens zijn herdefinieert. De Mens is niet langer een beest, onderdeel van de natuur, maar een een informatieprocessor, en de natuur zijn informatie. 
\end{document}