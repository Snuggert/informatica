\documentclass[pdftex,12pt,a4paper]{article}

\usepackage{wrapfig,amssymb,amsmath,graphicx, subfigure}
\usepackage[dutch]{babel}
\usepackage[top=0.5in, bottom=0.5in, left=1in, right=1in]{geometry}
\pagenumbering{arabic}
\newcommand{\HRule}{\rule{\linewidth}{0.5mm}}

\begin{document}
\bibliographystyle{plain}
\noindent Opdracht RDC5\\ 
Abe Wiersma(10433120)\\
\today
\section*{RDC5}
\subsection{Techniek Filosofie}
In het boek 'De grens van de Mens' praat Peter-Paul Verbeek over de kracht die de techniek op ons uitoefent. In zijn definitie van onze relaties met de techniek houdt hij de filosofie van Don Ihde aan, die in zijn boek 'Technology and the lifeworld' 4 basale relaties definieert tussen mens en techniek.
\begin{itemize}
\item De inlijvings-relatie waarbij de techniek deel gaat uitmaken van jezelf, denk aan een bril. 
\item De hermeneutische-relatie waarbij een techniek de wereld voor jou interpreteert, denk aan een thermometer.
\item De alteriteits-relatie waarbij er directe interactie is tussen den techniek en de mens, denk hierbij aan een dvd-speler of thermostaat.
\item De achtergrondsrelatie waarbij de techniek een rol speelt in de achtergrond van onze ervaring, denk aan de lampjes op je computer.
\end{itemize}
Vanuit deze vier relaties onderzocht Ihde hoe technologieën een rol spelen in de tot stand koming van interpretatie kaders. Met het onderzoek doorbrak Ihde de gevestigde techniek filosofie door te stellen dat men niet tegenover de techniek staat maar dat de techniek als bemiddelaar staat tussen mens en werkelijkheid.
\\
Verbeek stelt nu dat sinds het boek van Ihde dat in 1990 geoubliceerd werd er twee nieuwe relaties zijn ontstaan:
\begin{itemize}
\item De immersie-relatie, waarin de techniek zich om de mens heen vormt, denk aan slimme verlichting en slimme bedden.
\item De versmeltings-relatie, waarin de technologie de mens met behulp van de techniek wordt aangepast, denk aan implantaten en genetische aanpassingen.
\end{itemize}
Zeker door deze laatste relatie is goed te zien dat de grens tussen mens en techniek aan het verwateren is. Verbeek stelt dat het je willen zuiveren van technologie nutteloos is omdat je daarmee tegelijkertijd jezelf op zou heffen. Kapp, Schmidt en Gehlen stelden al dat de mens niet zou kunnen voortbestaan zonder de Techniek. Kapp was de eerste die veronderstelde dat de hamer een materiële projectie was van de organische vuist en zo verbond hij meerdere menselijke eigenschappen met de materiële techniek. Later in de $20^{e}$ eeuw werkte Schmidt deze relatie verder uit door drie stadia te onderscheiden in de ontwikkeling van de techniek waarin de beschrijvingen van Kapp de eerste fase beschreven. Deze eerste fase noemde Schmidt de fase van het werktuig. De tweede fase was de Machine fase waarin de techniek kracht ontleent aan zichzelf. De automaat is de derde fase waarin het menselijk subject niet meer benodigd is, voor het functioneren van de techniek zowel fysiek als intellectueel kan de automaat alles zelf af.

Later bouwde Arnold Gehlen voort op het werk van Kapp en Schmidt door wederom de vraag te stellen hoe techniek samenhing met de mens als organisch wezen. Deze drie relaties onderscheidde hij in zijn tekst 'A philosophical-anthropological perspective on technology':
\begin{itemize}
\item Orgaanvervanging, de hamer die de vuist verving.
\item Orgaanversterking, een microsoop die de mogelijkheden van het oog verbreid.
\item Orgaanontlasting, het wiel waarmee zware voorwerpen gemakkelijker verplaatst konden worden.
\end{itemize}
Met deze drie relaties merkte Gehlen op dat de organische structuur steeds verder werd vervangen door het anorganische.\\

Verbeek merkt op dat door de ontwikkelingen in de techniek aan de excentrische positionaliteit van Herlmuth Plessner een nieuwe betekenis gegeven kan worden. Plessner stelde dat omdat de mens excentrisch is, de mens van nature kunstmatig is. Dat de mens kunstmatig is komt door de spanning die excentriciteit creëert, dit komt omdat de mens zichzelf kan beschouwen en veronderstelt dat men niet gewoon 'bestaat'.

Vervolgens refereert Verbeek naar de tekst van Jos de Mul 'Cyberspace Odyssee waarin hij met behulp van de de moderne techniek komt tot een uitbreding van de excentriciteit. Door de virtuele werkelijkheden te vinden in de moderne techniek beschrijft de Mul hoe er te praten valt over een Poly-excentriciteit waarin we niet alleen de eigen wereld vanuit buiten beschouwt maar ook de virtuele.

Toch vindt Verbeek niet dat deze filosofie omvattend genoeg is omdat technologieën als psychofarmaca en genetische inteventies volgens hem hier niet inpassen. Dus stelt hij een nieuwe positionaliteit voor: de meta-excentriciteit. In deze excentriciteit verhoudt de mens zich niet alleen tot de werelden maar tot zijn eigen excentriciteit.
\end{document}