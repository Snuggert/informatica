\documentclass[pdftex,12pt,a4paper]{article}

\usepackage{wrapfig,amssymb,amsmath,graphicx, subfigure}
\usepackage[dutch]{babel}
\usepackage[top=0.5in, bottom=0.5in, left=1in, right=1in]{geometry}
\pagenumbering{arabic}
\newcommand{\HRule}{\rule{\linewidth}{0.5mm}}

\begin{document}
\bibliographystyle{plain}
\noindent Opdracht RDC1\\ 
Abe Wiersma(10433120)\\
\today
\section*{Software}
\subsection{Ada Lovelace (het begin)}
Ada Lovelace wordt door velen beschouwd als de eerste persoon die zich bezig
hield met het programmeren van computers. In haar aantekeningen, gepubliceerd
rond 1842, staat een algoritme beschreven dat met behulp van
'The Analytical Machine', bedacht door haar man Charles Babbage, Bernoulli
getallen produceert. De machine van Babbage is overigens nooit voltooid dus is
haar code nooit getest.
\subsection{De ontwikkeling}
In Mei 1949 werd de EDSAC(Electronic Delay Storage Automatic Calculator) door
de universiteit van Cambridge voltooid. EDSAC was de eerste computer die zijn
programma kon opslaan in zijn digitale geheugen. Met de EDSAC kwam ook de
realisatie dat het bouwen van een computer slechts stap 1 was, en dat om de
computer te gebruiken 'software' nodig was. Rond juni kwam men er op Cambridge
achter dat het ontwikkelen van routines(later programma's) verassend moeilijk
was. 

De programmeurs ontwikkelden een assembly manier voor het representeren van
operaties, zo kon een stukje code dat voor een additie van een
'short' getal in geheugen locatie 25 zorgde er zo uit zien als bitstring:
11100000000110010, en zo uit zien in 'assembly': A 25 S.\cite{campbell2014history}
De ingenieurs van Cambridge kwamen er al snel achter dat de computer deze
conversie afkon en op die manier niet in bit strings hoefden te programmeren.
In de jaren die volgden kwamen er woorden om het programmeerproces te
omschrijven, bijvoorbeeld: fouten werden bugs en het oplossen van fouten was
debugging. Er waren zoveel bugs in alle programma's die ontwikkeld werden dat
er een software crisis ontstond, programma's kostten te veel, deadlines werden
niet behaald en de eindresultaten zaten vol met bugs. De vraag waarom werkende
software maken zo moeilijk is houdt de afgelopen tientallen jaren de computer
industrie in zijn grip. Wanneer een bug zich voordeet had je geen traceback en
geen scherm om te zien waar je was gebleven in je code, wat ze wel hadden was
een luidspreker. Dus wanneer er een bepaalde toon net was geproduceerd wist je
waar de fout in de code was. Met de komst van de beeldschermen was dit niet meer
nodig en raakte luidspreker debugging in het verval.
Om bugs te verminderen werden vaak gebruikte functies in bibliotheken
gezet, hierdoor was het mogelijk dat programma uit twee derde
subroutines(functies die al geschreven waren) en één derde uit nieuwe code.

\subsection{Het resultaat}
Het vakgebied software engineering werd in 1967 voor het eerst officieel erkend
door een studie-groep van de NATO wetenschaps commissie, die opriep tot een
conferentie over het onderwerp.
\bibliography{bib}
\end{document}
