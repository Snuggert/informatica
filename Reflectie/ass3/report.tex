\documentclass[pdftex,12pt,a4paper]{article}

\usepackage{wrapfig,amssymb,amsmath,graphicx, subfigure}
\usepackage[dutch]{babel}
\usepackage[top=0.5in, bottom=0.5in, left=1in, right=1in]{geometry}
\pagenumbering{arabic}
\newcommand{\HRule}{\rule{\linewidth}{0.5mm}}

\begin{document}
\bibliographystyle{plain}
\noindent Opdracht RDC3\\ 
Abe Wiersma(10433120)\\
\today
\section*{RDC3}
\subsection{Het vakgebied informatica}
\subsubsection{Het begin}
In 1949 wordt op Cambridge de EDSAC voltooid, vanaf toen kwamen er wiskundigen die meer bezig waren met programmeren dan met de wiskunde, al hangt dit natuurlijk samen.
Deze 'programmeurs' waren samen met de computer-ingenieurs de eerste vertegenwoordigers van het vakgebied informatica.
Het vakgebied informatica vond in David Wheeler die in 1951 als een PhD behaalde in het 'nieuwe' vak Computer Science de eerste persoon die een studie rondom computers afrondde. Wheeler was verantwoordelijk voor het concept van de subroutine, een subroutine wordt nu een functie genoemd.
In Nederland duurde het een stuk langer voordat er officieel een informatica opleiding van de grond kwam. In 1969 werd al aan de Academische Raad van het Ministerie van Onderwijs en Wetenschappen gevraagd naar de mogelijkheden van een informatica opleding, het duurde nog 12 jaar tot de Academische Raad in 1981 besloot tot opnemen van informatica in Academisch Statuut. Dat het zo lang duurde kwam doordat de opleidingen het er niet over eens konden worden wat een opleiding informatica moest inhouden. Edsgar Dijkstra was bijvoorbeeld groot tegenstander van de individuele opleiding informatica omdat deze inhoudsloos zou zijn. Bestuurlijke informatica was al helemaal niks, dat was een "Cursus ponskaarten invullen".

Voor 1981 moesten studenten die ambieerden informaticus te zijn hun heil zoeken binnen de wiskunde opleidingen, en zelfs dan was je als je klaar was met je studie dan was je een wiskundige die een paar informaticavakken had gedaan. Dit alles staat natuurlijk in zwaar contrast met hoe de huidige informatica student een opleiding verlaat.
\subsubsection{Wat het nu is}
Ondertussen zijn jaren worsteling een groot aantal subdisciplines binnen de informatica ontstaan. De grootste hiervan zijn: de fundamentele informatica, informatie-en gegevensbeheer, kunstmatige intelligentie, software engineering, computernetwerken en mens-computerinteractie. Voor al deze disciplines is er aansluiting te vinden in de informatiewetenschappen op universiteiten.
Ondanks het ruime aanbod in informatica studies zijn er veel te weinig ict-ers. Nederland ICT schat dat er in 2017 een tekort van 6800 ict-ers zal zijn. Dit aantal zou zelfs nog kunnen oplopen tot de 12000 wanneer de economie zou aantrekken. Om dit op te lossen wordt er veel gedaan aan de bevordering van het imago van de informatica. Er zijn veel campagnes om vrouwen aan te trekken omdat de verdeling man/vrouw uiterst ongelijk is en er wordt gepoogd informatica van zijn nerd-image af te helpen. De opkomst en beschikbaarheid van consumententechnologiën in de samenleving heeft gezorgd voor een transitie van achterkamertjes-geklungel tot eettafel-apparatuur. Tegenwoordig heeft jong en oud affiliaties met de informatica door allerhande apparatuur, en hierdoor is het stigma nerd langzaamaan aan het verdwijnen. Ik denk dat naarmate het stigma meer verdwijnt zo ook de tekorten aan informatici zullen verdwijnen. 
\subsubsection{Wat ik met mijn studie informatica kan}
Ik denk dat een opleiding in de informatica meestal geen rechte weg naar één baan biedt. Dit komt omdat het vak informatica veel te breed is, systeemonderhoud wat zeker onder de informatica valt is iets totaal anders dan webdevelopment. Dit zijn banen die je gemakkelijk na je studie kan doen, maar waar weinig aandacht aan is gegeven in je opleiding. In de informatica wordt je zo breed opgeleid dat je met alles wat met computers te maken heeft een makkelijke link kan leggen naar een vak dat je gehad hebt. Op die manier is de informatiewetenschappelijke discipline vooral een hele brede basis om van alle kanten gebruikt te kunnen worden.
\end{document}