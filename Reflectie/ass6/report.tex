\documentclass[pdftex,12pt,a4paper]{article}

\usepackage{wrapfig,amssymb,amsmath,graphicx, subfigure}
\usepackage[dutch]{babel}
\usepackage[top=0.5in, bottom=0.5in, left=1in, right=1in]{geometry}
\pagenumbering{arabic}
\newcommand{\HRule}{\rule{\linewidth}{0.5mm}}

\begin{document}
\bibliographystyle{plain}
\noindent Opdracht RDC6\\ 
Abe Wiersma(10433120)\\
\today
\section*{RDC5}
\subsection{Techniek en verantwoordelijkheid}
\subsubsection{De (huidige) ethiek van Hamermas}
Hoe de ethiek een rol speelt in de techniek filosofie probeert Peter-Paul Verbeek te beantwoorden in het laatste hoofdstuk van zijn boek 'De grens van de mens'. Verbeek legt eerst het standpunt uit van Habermas uit die de techniek en de mens als twee sferen ziet en de ethiek als bewaker tussen deze twee sferen. Deze scheiding tussen de twee sferen techniek en de mens bestaat volgens Verbeek niet. Volgens Verbeek zijn de techniek en de mens onhoudbaar verwoven met elkaar.  Als voorbeeld verwijst hij naar de immersie- en versmeltingsrelatie die hij in voorgaande hoofdstukken had uitgewerkt. Deze moderne relaties die de techniek hebben met de mens hebben een wisselwerking met zowel de existentiële vorm als de biologische vorm van ons leven.
Waar Habermas stelt dat de ethische grens van de techniek ligt bij de eugenetica omdat daarmee de individualiteit van de mens wordt afgenomen. Verbeek stelt echter dat de ethische grens verder ligt, dit omdat volgens Verbeek het overstijgen van grenzen de mens eigen is.
\subsubsection{De ethiek van Verbeek}
Als alternatief stelt Verbeek een nieuwe ethiek voor, een ethiek waarin geen uitersten worden opgezocht en de sferen mens en techniek niet bestaan. In de techniek valt niet blind te vertrouwen zoals de transhumanisten dat doen en ook volledig wantrouwen zoals de bioconservatisten moet vermeden worden. Inplaats hiervan moet een ethiek ontwikkeld worden die ten doel stelt de mens toe te vertrouwen aan de techniek. Deze ethiek kan volgens Verbeek beter omschreven worden als technologie begeleidend in plaats van het bewaken van de grens tussen mens en techniek zoals Habermas' ethiek. Daar de techniek vanuit deze ethiek niet meer zwart op wit tegen over de mens staat is het gebruik van technologie meer een overweging dan het opzoeken van een grens en de eventuele overschreiding daarvan.
\end{document}