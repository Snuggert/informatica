\documentclass[pdftex,12pt,a4paper]{article}

\usepackage{wrapfig,amssymb,amsmath,graphicx, subfigure}
\usepackage[dutch]{babel}
\usepackage[top=0.5in, bottom=0.5in, left=1in, right=1in]{geometry}
\pagenumbering{arabic}
\newcommand{\HRule}{\rule{\linewidth}{0.5mm}}

\begin{document}
\begin{titlepage}

% Upper part of the page
\begin{flushleft}
\includegraphics[trim=23mm 0mm 0mm 0mm, width=1\textwidth]{./logo.jpg}\\[1cm] \end{flushleft}
\begin{center}
	\textsc{\Large Reflectie op de Digitale Cultuur}\\[0.5cm]

    % Title
    \HRule \\[0.4cm] { \huge \bfseries Geosociale dating-apps}\\[0.4cm]

    \HRule \\[1.5cm]
    RDC-W3

    % Author and supervisor
\begin{minipage}{0.4\textwidth}
\begin{flushleft} \large \emph{Authors:}\\
Abe \textsc{Wiersma}\\
\end{flushleft}
\end{minipage}
\begin{minipage}{0.4\textwidth} \begin{flushright} \large \end{flushright}\end{minipage}

    \vfill

    % Bottom of the page 
    {\large \today}

\end{center}
\end{titlepage}
\pagebreak
\bibliographystyle{plain}
\section{Context}
De opkomst van geosociale dating-apps vindt zijn oorsprong in Grindr, de homo dating-app die werd uitgebracht in maart 2009. Er waren snel veel positieve reviews en na twee jaar, op de 'verjaardag' van de app, waren er een totaal van 1.7 miljoen gebruikers in 190 landen. Met het succes van Grindr in het achterhoofd werd, een half jaar na het tweejarig bestaan, in september 2011 de nieuwe app Blendr gelanceerd. Joel Simkhai begon met Blendr nadat hij overwelmt werd door hetero-vrouwen die 'jaloers' waren op hun homo-vrienden die Grindr hadden. Toch is er volgens hem een groot verschil tussen Blendr en Grindr, dit omdat Blendr meer gericht zou zijn op vriendschap en minder op sex.
\\

Dan in september 2012 wordt Tinder gelanceerd, eerst op de Universiteit van Zuid-Californië, daarna voor andere universiteiten en uiteindelijk voor iedereen. Tinder lijkt erg op Blendr in het feit dat de apps beide gebruik maken van het Facebook account van de gebruiker om informatie op te halen.
\\

Op het moment is er zelfs een app genaamd 3nder onder ontwikkeling die voorziet in communicatie tussen 3 mensen, deze app is bedoeld om stellen in contact te brengen met een willende derde of om 3 mensen aan elkaar te koppelen.

\section{Statistieken}
Waar Grindr heel open is over het aantal gebruikers wereldwijd, blijven die gegevens bij Tinder en Blendr achter slot en grendel. Het is al helemaal vreemd omdat Blendr en Grindr door dezelfde persoon zijn gemaakt. Hoewel beide apps in hun privacy statements hebben staan dat gegevens als aantal gebruikers gepubliceerd mogen worden, doen ze dat niet. Waar Grindr trots durft te vertellen dat ze 4 miljoen mannen als gebruikers hebben, laat Blendr weten dat ze erg blij zijn dat de groei in het eerste half jaar groter was dan die van Grindr.\cite{creeped2013} Na contact met Tinder gaven ze mij, net als mijn voorgangers, geen antwoord op mijn vraag hoeveel gebruikers ze hebben.

Toch kon Tinder mij vertellen dat ze al 1.5 Miljard matches hadden gemaakt, waarbij een match een positieve rating van 2 gebruikers aan elkaar is. Vanaf dit moment kan men in contact met elkaar komen. 

Te concluderen valt is dat  een substantiële grootte van de samenleving zich bezig houdt met de online date ervaring.

\section{Invloed op samenleving}
Ter illustratie van mijn punt wil ik graag het Tinder-gebruik van een goede vriend van mij uit de doeken doen. Hij heeft Tinder al zo'n anderhalf jaar en verteld trots dat hij al een enorme hoeveelheid matches heeft, meer dan 300. Hoewel hij dus blijkbaar gewild is heeft hij nog nooit een date gehad met één van zijn matches. In die tijd heeft hij buiten het 'virtuele' wel dates gehad en zonder mediatie van een dating-app. Mijn conclusie hieruit is dat de dating-app in dit geval niet echt gebruikt word om contact te vinden met gelijkgestemden maar ter verrijking van zijn ego. Het behalen van een groot aantal matches zal misschien een leuk feitje zijn om laten zien hoe gewild je bent op het internet, maar meer dan dat zal het niet zijn.\\
Toch vermoeilijkt hij contact met potentiële dates wanneer hij in plaats van socializen zijn telefoon opzoekt waar zoals blijkt slechts date-gespreksstof te halen valt en geen dates.\\

Het verhaal van mijn vriend is niet uniek, het komt  zelfs zo vaak voor dat ik denk dat ik denk dat de hetero geosociale dating apps geen toegevoegde waarden hebben in de samenleving.

\section{Conclusie}
Hoewel dating-apps taboedoorbrekend begonnen met Grindr, de gaydar van Joel Simkhai, kwam er in mijn ogen een eind aan de functionaliteit toen de hetero versies werden geïntroduceerd. Waar Grindr zorgde voor een 'veilig' platform voor een relatievorm die nog steeds word gestigmatiseerd, zorgden Tinder en Blendr voor relaties die online begonnen en online eindigden. Nu ook 3nder de strijd aan wil gaan met stigmatisering gloort er misschien hoop aan de horizon voor de geosociale dating-apps. 
\bibliography{lib}
\end{document}